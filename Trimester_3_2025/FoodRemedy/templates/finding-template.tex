\documentclass[12pt]{article}

\usepackage[margin=2.5cm]{geometry}
\usepackage{hyperref}
\usepackage{graphicx}

% ===========================
% AppAttack Finding Template
% ===========================

% Naming convention reminder:
% NAME_FINDINGTYPE_TITLE
% Example:
% Maaz_PT_SQLInjection
% Alice&Bob_SCR_InsecureDeserialization

\begin{document}

% ---------- Metadata ----------
\title{[PROJECT NAME] - Vulnerability Finding}
\author{[Your Name]}
\date{\today}

\maketitle

\section*{Finding ID}
% Example: FR-001 or APPATTACK-FR-2025-001
[Enter a unique ID]

\section*{Finding Title}
[Short title, e.g. SQL Injection in Login Form]

\section*{Type}
% PT = Penetration Testing, SCR = Secure Code Review
[PT or SCR]

\section*{Summary}
[One short paragraph summarising the issue in plain language.]

\section*{Technical Description}
[Describe the vulnerability in detail. Where is it? How does it work?]

\section*{Steps to Reproduce}
\begin{enumerate}
  \item Step 1...
  \item Step 2...
  \item Step 3...
\end{enumerate}

\section*{Evidence}
[Describe screenshots, logs, or requests/responses. 
Include \texttt{\textbackslash includegraphics} if you want:]

% \begin{figure}[h]
%   \centering
%   \includegraphics[width=0.9\textwidth]{screenshots/example.png}
%   \caption{Evidence of SQL injection error message}
% \end{figure}

\section*{Impact}
[Explain the potential damage: data leakage, account takeover, etc.]

\section*{Risk Rating}
% You can later standardise this (e.g. CVSS or simple High/Medium/Low)
[High / Medium / Low with a short justification.]

\section*{Recommendation}
[How should the devs fix it? Input validation, parameterised queries, etc.]

\section*{References}
[Links to OWASP pages, docs, etc.]

\end{document}
